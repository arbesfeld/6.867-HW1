%% skeleton.tex                 %% 12 February 2006
%% Use this file to start your paper for the UJM/18.096

\documentclass[11pt]{article}   %% Standard LaTeX.
\usepackage{amsmath,amssymb}    %% For better support of math
                                %% amssymb provides \mathbb and \square
                                
\usepackage{xcolor}
%% \usepackage{url}             %% Supports formating URLs.
%% \usepackage{graphicx}        %% Enable for eps figures

\newcommand\note[1]{\textcolor{red}{#1}}
\newcommand\PP{\mathbb{P}}
\newcommand\seq{\textsc{seq}_l}
\newcommand{\qed}{\hfill \ensuremath{\Box}}

\textwidth=6.8in
\textheight=8in
\hoffset=-0.85in
\topmargin=0.5in
\setlength{\topmargin}{0in}
\usepackage{ntheorem}
\usepackage{graphicx}

\theoremstyle{plain}
\theorembodyfont{}
\theoremsymbol{}
\theoremprework{}
\theorempostwork{}
\theoremseparator{.}

\newtheorem{theorem}{Theorem}[section]
\newtheorem{lemma}[theorem]{Lemma}
\newtheorem{proposition}[theorem]{Proposition}
\newtheorem{corollary}[theorem]{Corollary}

\newenvironment{proof}[1][Proof.]{\begin{trivlist}
\item[\hskip \labelsep {\bfseries #1}]}{\end{trivlist}}
\newenvironment{definition}[1][Definition.]{\begin{trivlist}
\item[\hskip \labelsep {\bfseries #1}]}{\end{trivlist}}
\newenvironment{example}[1][Example.]{\begin{trivlist}
\item[\hskip \labelsep {\bfseries #1}]}{\end{trivlist}}
\newenvironment{remark}[1][Remark.]{\begin{trivlist}
\item[\hskip \labelsep {\bfseries #1}]}{\end{trivlist}}
\newcommand\numberthis{\addtocounter{equation}{1}\tag{\theequation}}

\begin{document}
\pagestyle{myheadings}          %% Supports custom headers.
\markboth{\sc 6.837 - HW1}{\sc
6.837 - HW1}                  %% Running right header
\title{6.837 - Homework 1}           %% For first page
\author{Akhil Raju and Matthew Arbesfeld}
\date{September 28, 2014}         %% Change \today to draft date if you want
\maketitle


\section{Introduction}\label{sec-intro}
The inclusion-exclusion principle is one of the most important results of probability theory. If \textbf{A} is the union of events $A_1, A_2,\ldots,A_n$, then the inclusion-exclusion principle gives us $\PP(\textbf{A})$ from $\PP(A_i), \PP(A_i \wedge A_j), \PP(A_i \wedge A_j \wedge A_k)$ and all other intersections of the subsets of $A$. \\
\\
\indent In probability theory, we often have information about the probability of individual events as well as the probabilities of their intersections. For example, suppose that we wish to find the probability of flipping two fair coins and getting \textit{at least one} heads. If $H_1$ is the event that the first coin is heads and $H_2$ is the event that the second coin is heads, then the inclusion-principle tells us that
\begin{align*}
\PP(H_1 \vee H_2) = \PP(H_1) + \PP(H_2) - \PP(H_1 \wedge H_2) = \frac{1}{2} + \frac{1}{2} - \frac{1}{4} = \frac{3}{4}.
\end{align*}
Thus, we can determine the union of the set of events from their individual probabilities and intersection probabilities. \\
\\
\indent To use the inclusion-exclusion principle on a set of events $A_1, A_2,\ldots,A_n$, we must know the probabilities of the intersections for all subsets of the events:
\begin{align*}
\forall i \in \{1,2,\ldots,n\}&: \PP(A_i) \\
\forall (i,j) \in \{1,2,\ldots,n\}, i \ne j&: \PP(A_i \wedge A_j) \\
\forall (i,j,k) \in \{1,2,\ldots,n\}, i \ne j \ne k&: \PP(A_i \wedge A_j \wedge A_k) \\
&\vdots \\
\forall (x_1, x_2, \ldots, x_n) \in \{1,2,\ldots,n\}, x_1 \ne x_2 \ne \ldots \ne x_n&: \PP(A_{x_1} \wedge A_{x_2} \wedge \ldots \wedge A_{x_n}). \\
\end{align*}
In practice, though, we may not have information on all of these intersections. For example, we may only know the probability of individual events and the probability of pairwise intersections of events. In this paper, we will develop an extension of the inclusion-exclusion principle for cases when we are given only a subset of the intersection probabilities. Specifically, we will find upper and lower bounds on the probability of the union of $n$ events when given the intersection probabilities for all subsets of up to $k$ events with $k < n$. These bounds are known as the Bonferroni inequalities. A more comprehensive analysis on inclusion-exclusion with limited information can be found in Linial and Nisan's work \cite{linial}. \\
\\
\indent In Section~\ref{sec:inclusion-exclusion}, we formally introduce the inclusion-exclusion principle and define notation that we will use for representing the probability for intersections of events. In Section~\ref{sec:bounds}, we develop an extension of the inclusion-exclusion principle which gives us bounds on the union of a set of events when we have knowledge on only a subset of the intersection probabilities. Finally, in Section~\ref{sec:coins}, we use our extension of the inclusion-exclusion principle to find the probabilistic bounds of a flip sequence in a coin-flipping game.

\begin{thebibliography}{9}  %% Change to 99 if you have 10 to 99 entries

%% These are sample bibliography entries, and should of course be deleted.
%% But follow the same style; in particular, alphabetize them by
%% author's last name.
\bibitem{linial}
N. Linial and N. Nisan, ``Approximate Inclusion-Exclusion,'' Combinatorica 10, (1990) 349-365.

\end{thebibliography}

\end{document}
